\documentclass[preprint,12pt]{article}
\usepackage[top= 2cm, bottom = 2 cm, left = 2.5 cm, right = 2.5 cm]{geometry}
\usepackage[UTF8]{inputenc}
\setlength{\parindent}{0 pt}
\pagestyle{empty}
\title{SQL: 2016 IS THE ISO/IEC 9075:2016 STANDARD OF 2016}
\author{Pedro Luis Mamani Mamani (2010038808)}
\usepackage[spanish]{babel}
\usepackage[T1]{fontenc}
\usepackage[document]{ragged2e}
\usepackage{graphicx}


\begin{document}
\maketitle

\rule{170mm}{0.1mm}
\begin{justify}
\textbf{Abstract}  \\\\ SQL:2016 or ISO/IEC 9075:2016 (under the general title "Information technology – Database languages – SQL") is the eighth revision of the ISO (1987) and ANSI (1986) standard for the SQL database query language. \\\\ It was formally adopted in December 2016.[1] The standard consists of 9 parts which are described in some detail in SQL. 
\end{justify}

\rule{170mm}{0.1mm}
\begin{justify}
\textbf{1.- Resumen}  \\\\ SQL: 2016 o ISO / IEC 9075: 2016 (bajo el título general "Tecnología de la información - Lenguajes de base de datos - SQL") es la octava revisión del estándar ISO (1987) y ANSI (1986) para el lenguaje de consulta de la base de datos SQL. \\\\ Fue adoptado formalmente en diciembre de 2016. [1] El estándar consta de 9 partes que se describen con cierto detalle en SQL.
\end{justify}


\begin{justify}
\textbf{2.- Introducción}  \\\\ 
SQL (siglas en inglés de Structured Query Language) es el lenguaje estándar para los sistemas de gestión de bases de datos relacionales. Cuando se originó en la década de 1970, el lenguaje específico del dominio estaba destinado a satisfacer la necesidad de realizar una consulta en la base de datos que pudiera navegar a través de una red de punteros para encontrar la ubicación deseada. Su aplicación en el manejo de datos estructurados se ha fomentado en la era digital. De hecho, las poderosas capacidades de manipulación y definición de bases de datos de SQL y su vista tabular intuitiva han estado disponibles de alguna forma en prácticamente todas las plataformas informáticas importantes del mundo.\\\\
\end{justify}

\begin{justify}
\textbf{3.- Marco Teórico}  \\\\ 
El Estándar ANSI para SQL No existen los Estándares ANSI , ya que ANSI no desarrolla estándares. En cambio, hay estándares nacionales estadounidenses y otros documentos escritos por comités de organizaciones de desarrollo de estándares aprobados por ANSI. No obstante, recibimos muchas consultas sobre el "estándar ANSI" para SQL. Vale la pena señalar que, si bien esta frase es engañosa e inexacta por numerosas razones, se refiere a documentos estándar existentes. SQL, al igual que muchas cosas geniales que sobrevivieron a los años 70, tiene una historia poderosa entrelazada auspiciosamente con los estándares. En la concepción de las especificaciones de SQL estaba ANSI (solo otra gran hazaña a lo largo de nuestros 100 años de historia).\\\\La edición actual de ISO / IEC 9075 para SQL Si no desea leer todo el historial de SQL que hemos detallado a continuación, para abreviar, SQL se estandarizó en ANSI X3.135 en 1986 y, dentro de un pocos meses, fue adoptado por ISO como ISO 9075-1987. Si bien la mayoría de los proveedores modifican SQL para satisfacer sus necesidades, generalmente basan sus programas en la versión actual de este estándar. El estándar internacional (ahora ISO / IEC 9075) se ha revisado periódicamente desde entonces, más recientemente en 2016. Existe en 9 partes:
	
\begin {enumerate}

\item ISO/IEC 9075-1:2016 – Information technology – Database languages – SQL – Part 1: Framework (SQL/Framework) 

\item This document has a corrigendum: ISO/IEC 9075-1:2011/Cor1:2013 
ISO/IEC 9075-2:2016 – Information technology – Database languages – SQL – Part 2: Foundation (SQL/Foundation) 

\item ISO/IEC 9075-3:2016 – Information technology – Database languages – SQL – Part 3: Call-Level Interface (SQL/CLI) 

\item ISO/IEC 9075-4:2016 – Information technology – Database languages – SQL – Part 4: Persistent stored modules (SQL/PSM) 

\item ISO/IEC 9075-9:2016 – Information technology – Database languages – SQL – Part 9: Management of External Data (SQL/MED) 

\item ISO/IEC 9075-10:2016 – Information technology – Database languages – SQL – Part 10: Object language bindings (SQL/OLB) 

\item ISO/IEC 9075-11:2016 – Information technology – Database languages – SQL – Part 11: Information and definition schemas (SQL/Schemata)

\item ISO/IEC 9075-13:2016 – Information technology – Database languages – SQL – Part 13: SQL Routines and types using the Java TM programming language (SQL/JRT) 

\item ISO/IEC 9075-14:2016 – Information technology – Database languages – SQL – Part 14: XML-Related Specifications (SQL/XML)\\\\


\end {enumerate}
	

\end{justify}

\begin{justify}
\textbf{4.- Analisis} \\\\

\textbf {SQL: A Standardized History}\\\\

Cuando se estableció SQL a principios de los años 70, se llamaba SEQUEL (Lenguaje de consulta estructurado en inglés). Sin embargo, debido a un problema de derechos de autor, se cambió a SQL. De hecho, SQL se pronuncia "secuela" en la actualidad, pero algunos prefieren la pronunciación sin acrónimos de "ess-cue-el" (en caso de que no lo supiera, un acrónimo es una abreviatura que puede pronunciar como una palabra, p. Ej. SQL o ANSI).\\\\Los progenitores de SQL, Donald Chamberlin y Ray Boyce, basándose en el modelo establecido por EF Codd en su artículo, "Un modelo relacional de datos para grandes bancos de datos compartidos", desarrollaron el lenguaje de programación en su propio documento "SEQUEL: A Structured English Query Language ”. Tomaron los idiomas de Codd con el objetivo de diseñar un lenguaje relacional que fuera más accesible para los usuarios sin una capacitación formal en matemáticas o ciencias de la computación. Esta versión SQL original que diseñaron se utilizó para manipular y recuperar datos almacenados en los sistemas de bases de datos relacionales originales de IBM, conocidos como "Sistema R".
\\\\En los años siguientes, SQL no estaba disponible públicamente. Sin embargo, en 1979, Oracle, entonces conocido como Software Relacional, lanzó su propia versión de SQL llamada Oracle V2, que se lanzó comercialmente. Es importante tener en cuenta que SQL no fue el primer lenguaje de programación para navegar por las bases de datos, pero cortó una presencia impecable debido a su intuición, potencia y confiabilidad. Hay una razón por la que todavía estamos hablando de eso hoy. Sin embargo, el éxito prolongado de SQL no se puede atribuir solo a sus cualidades. La ayuda de los estándares no solo ayudó a SQL a acercarse a la universalidad, sino que también agregó atributos clave a lo que ha florecido en sus especificaciones modernas. Todo esto comenzó cuando ANSI se involucró.\\\\En el momento de la publicación inicial de este estándar, seguramente se necesitaban especificaciones más profundas con respecto a SQL, pero ANSI X3.135-1986 ayudó a establecer las bases para algunas capacidades importantes para el lenguaje de codificación. Este estándar dio la posibilidad de invocar capacidades SQL desde cuatro lenguajes de programación: COBOL, FORTRAN, Pascal y PL / I. Estas normas fueron revisadas en concierto, primero en 1989 (ANSI X3.135-1989 e ISO / IEC 9075: 1989) y nuevamente en 1992 (ANSI X3.135-1992 e ISO / IEC 9075: 1992). La edición de 1989 agregó soporte para dos lenguajes de programación adicionales, Ada y C. Estas ediciones se conocieron coloquialmente como SQL-86, SQL-89 y SQL-92. Entonces, si escucha esos nombres en referencia a un formato SQL, tenga en cuenta que se refiere a las diversas ediciones de este estándar.

			\begin{center}
					\includegraphics[width=15cm]{./img/1}
				\end{center}	

Antes de la próxima revisión, el Comité de Estándares Acreditados X3, Tecnología de la Información, cambió. De 1961 a 1996, este comité de desarrollo de normas acreditado por ANSI funcionó en numerosas facetas de la industria mientras estaba patrocinado por ITI, una asociación comercial que entonces se conocía como la Asociación de Equipos de Computación y Negocios (CBEMA).\\\\Sin embargo, al final de este período. ASC X3 se convirtió en INCITS (el Comité Internacional de Estándares de Tecnología de la Información), una organización de desarrollo de estándares acreditada por ANSI. El estándar fue revisado nuevamente en 1993 (SQL3), 2003, 2008, 2011 y 2016, que sigue siendo la edición actual. El estándar SQL ha venido en varias partes desde el cambio de siglo, pero, desde la edición de 2003, se ha subdividido en 9 partes, cada una de las cuales cubre un aspecto diferente del estándar general y se incluye en el título que abarca, Tecnología de la información - Base de datos idiomas - SQL.\\\\La norma internacional ISO / IEC 9075 para SQL es desarrollada por el Comité Técnico Conjunto (JTC) 1 de ISO / IEC para Tecnología de la Información. La edición actual, ISO / IEC 9075: 2016, ha tenido cada una de sus 9 partes adoptadas por INCITS como estándares nacionales estadounidenses. Este proceso de desarrollo de medio siglo nos trajo una forma más viable del lenguaje SQL que conocemos y de la que dependemos hoy. Muchos proveedores utilizan el SQL y, aunque la mayoría de los principales vendedores modifican el lenguaje para satisfacer sus deseos, la mayoría basa sus programas SQL en la versión estándar. ISO / IEC 9075: 2016, al igual que muchos otros estándares de consenso voluntario existentes, está diseñado para fomentar la innovación y la competitividad, no obstaculizarla.

\begin{center}
					\includegraphics[width=15cm]{./img/2}
				\end{center}	

\end{justify}


\begin{justify}
\textbf{5.- Conclusiones}  \\\\ 
 - SQL: 2016 introdujo 44 nuevas características opcionales. [2] 22 de ellos pertenecen a la funcionalidad JSON, diez más están relacionados con funciones de tabla polimórficas. Las adiciones al estándar incluyen:\\\\- JSON: funciones para crear documentos JSON, acceder a partes de documentos JSON y verificar si una cadena contiene datos JSON válidos
Reconocimiento de patrón de fila: hacer coincidir una secuencia de filas con un patrón de expresión regular
Formato y análisis de fecha y hora
\\\\- LISTAGG: una función para transformar valores de un grupo de filas en una cadena delimitada
Funciones de tabla polimórficas: funciones de tabla sin tipo de retorno predefinido
\\\\- Nuevo tipo de datos DECFLOA

\end{justify}


\begin{justify}
\textbf{6.- Referencias}  \\\\ 
 {[1] "ISO/IEC 9075, 13249, and others". Retrieved 2017-03-15.}\\\\
 {[2] "What's New in SQL:2016". Retrieved 2017-06-16.}
\end{justify}


\end{document}